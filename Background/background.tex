\chapter{Mathematical background} \label{background}
In this chapter, we will provide some essential mathematical backgrounds required for this thesis. Some of these contents were included in previous works, for example, conformal welding part was first introduced in \cite{parallel} and \cite{Yusan}, and the contents in quasi-conformal map can be traced back to \cite{nDregistration}.

\section{Conformal maps}
Conformal maps are nice in preserving geometric properties, in a way that the angles are well-preserved under such transformations, geometrically transforming infinitesimal circles to infinitesimal circles.

Mathematically, let $U\subset \C$ be any open subset. Then $f: U\rightarrow\C$ is conformal if $f$ is holomorphic and bijective. By definition of being holomorphic, $f = u+iv$ satisfies the Cauchy-Riemann equations:
\begin{align}
    \dfrac{\partial u}{\partial x} &= \dfrac{\partial v}{\partial y} \nonumber\\
    \dfrac{\partial u}{\partial y} &= -\dfrac{\partial v}{\partial y}
    \label{eqn: CauchyRiemann}
\end{align}
This is equivalent to having 
\begin{align*}
    \dfrac{\partial u}{\partial x} - \dfrac{\partial v}{\partial y} = 0,\hspace{3pt} \dfrac{\partial u}{\partial y}+\dfrac{\partial v}{\partial x} = 0
\end{align*}{}
Consequently, conformal maps can be obtained by minimizing the following energy:
\begin{equation}
    E_C(f)=\dfrac{1}{2}\int_U(\dfrac{\partial u}{\partial x}-\dfrac{\partial v}{\partial y})^2 + (\dfrac{\partial u}{\partial y}+\dfrac{\partial v}{\partial x})^2dA
    \label{eqn: conformal energy}
\end{equation}{}
To see that conformal maps preserves angles, let $\gamma_i: [-\epsilon,\epsilon]\rightarrow U, \epsilon>0$ with $i=1,2$ be two curves satisfying that $\gamma_i(0) = z$ with $\gamma_i'(0) = v_i, i=1,2$. The angle preserving property in the case on complex plane can be easily verified by:
\begin{equation}
    \dfrac{(f\circ\gamma_1)'(0)(f\circ\gamma_2)'(0)}{(\|f\circ\gamma_1)'(0)(f\circ\gamma_2)'(0)\|} = \dfrac{[f'(z)]^2\gamma_1'(0)\gamma_2'(0)}{\|f'(z)\|^2\|\gamma_1'(0)\gamma_2'(0)\|} = \dfrac{v_1v_2}{\|v_1v_2\|}.
\end{equation}
There are a lot of conformal maps in practice, we will introduce an important kind of them, namely Mobi\"us transformation here.

\section{Mobi\"us transformation}
A Mobi\"us transformation, is one kind of linear fractional transformation on the (extended) complex plane $M: \C \rightarrow \C$, with the general expression:
\begin{equation}
    M(z) = \dfrac{az+b}{cz+d}
\end{equation}{}
$a,b,c,d \in \C$ with $ad-bc\neq 0$ for the invertibility and hence conformality. 

In the fact, the automorphism of the unit disk $\D$ is special kind of Mobi\"us transformations, which can be written as:
\begin{equation}
    M(z) = e^{i \theta} \frac{z - a}{1 - \overline{a}z}.
\end{equation}
where $a \in \D$, $\theta \in [0, 2\pi)$.

There are a few key properties of Mobi\"us transformation: It has typical correspondence of points: $(0, -\frac{d}{c}, \infty)\mapsto(\frac{b}{d}, \infty, \frac{a}{c})$ whenever these points are well-defined on the extended complex plane. Besides, all conformal maps are known to be equivalent up to a Mobi\"us transformation.

And for this reason, to obtain a unique conformal map, one still needs to provide a three-point correspondence. Also, later on we will make use of Mobi\"us transformation in many place.

\section{Conformal welding}
Conformal welding is a problem that occurs in complex analysis: Given 2D bounded simply-connected shape, we can treat it as a boundary simply-connected domain $\Omega \subset \C$, by Riemann mapping theorem, there exist conformal function $\Phi_1: \D \rightarrow \Omega$ and $\Phi_2: \D^c \rightarrow \Omega^c$, where $\D^c $ and $\Omega^c$ are the exterior part of $\D, \Omega$ respectively. $\Phi_1$ and $\Phi_2$ are unique up to a Mobi\"us transformation.

Then we can define \textit{conformal welding} as:
\begin{equation}
    f = \Phi_1^{-1} \circ \Phi_2.
\end{equation}
Such $f: \partial \D \rightarrow \partial \D$ is a diffeomorphism from $\partial \D$ to itself, which can be also thought as a periodic, real-valued monotone increasing function $f_\R: [0, 2\pi) \rightarrow [0, 2\pi)$ such that $f(e^{i \theta}) = e^{i f_\R(\theta)}$ (see Fig. \ref{ill of cw}). 

Conformal welding is a kind of contour-based shape representation but it's easy to find that such welding mapping $f$ is not unique because of the arbitrariness of Riemann mappings $\Phi_1$ and $\Phi_2$, as shown in Fig. \ref{unsatble}.

On the other side, for general homeomorphism $f$, there may be no domain $\Omega$ such that $f$ is just the corresponding Conformal welding. However, if $f$ satisfies certain conditions, this problem is solvable. Before that we need to define \textit{quasisymmetric function}:

\begin{definition}
    Let $f$ be a continuous, strictly increasing function defined on an interval $I$, we call $f$ is quasisymmetric on $I$ if there exists a positive constant $k$ such that
    \begin{equation}
        \frac{1}{k} \le \frac{f(x+t) - f(x)}{f(x)-f(x-t)} \le k
    \end{equation}
    for all $x, x-t, x+t \in I$ with $t>0$.
\end{definition}

Pfluger proved the following Sewing theorem based on the existence of solution to the Beltrami equation \cite{pfluger1960ueber}. This theorem tells us if $f$ is a quasisymmetric function, we can reconstruct the original domain as well as the conformal mappings.

\begin{theorem}[Sewing theorem]
    Let $f$ be a quasisymmetric function, then there is a Jordan domain $\Omega$ and conformal mappings $\Phi_1: \D \rightarrow \Omega$ and $\Phi_2: \D^c \rightarrow \Omega^c$ such that 
    \begin{equation}
        f = \Phi_1^{-1} \circ \Phi_2.
    \end{equation}
    The domain $\Omega$ is unique up to Mobi\"us transformations.
\end{theorem}

\begin{figure}
    \begin{center}
        \includegraphics[width=7cm]{./src/unstable_welding.png}
    \end{center}
    \caption{Different conformal welding mappings (b), (c) and (d) of the same shape (a)}
    \label{unsatble}
\end{figure}

\section{Harmonic maps}
A complex-valued map $f$ defined on $\Omega \subset \C$ is called \textit{harmonic} if it satisfies the \textit{Laplace's equation}:
\begin{equation}
    \Delta f = 4 \frac{\partial^2 f}{\partial z \partial \overline{z}} = \frac{\partial^2 f}{\partial x^2} + \frac{\partial^2 f}{\partial y^2} = 0,
\end{equation}
where $z = x + iy$, $x$, $y$ is the real and imaginary value.

Chen \etal prove following theorem in \cite{chen2010compositions}, which tells us that the composition of harmonic mappings and other mappings can inherit such harmonicity in some condition.
\begin{theorem}\label{composition of harmonic}
    Let $f$ be a harmonic mapping, $f \circ g$ is harmonic if and only if $g(z) = az + b \overline{z} + c$, where $a$, $b$ and $c$ are constants and $g \circ f$ is harmonic if and only if $g$ is analytic or anti-analytic.
\end{theorem}

When $\Omega$ is a compact set, the harmonic function $f$ defined on it satisfies the following maximum principle:
\begin{theorem}[Maximum principle]\label{Maximum principle}
    If $\Omega$ is a nonempty compact subset of $\C$, then harmonic function $f: \Omega \rightarrow \C$ attains its maximum on the boundary of $\Omega$.
\end{theorem}
If we apply maximum principle to $-f$, we can know that the minimun also must be on the boundary of $\Omega$. Futhermore, $f$ cannot have local maxima or minima, other than the exceptional case where $f$ is constant.

\section{Dirichlet problem and Poisson integral}
A harmonic function on a compact set is determined by its restriction to the boundary. The progress of finding a harmonic function defined on the given domain with the same value with a given continuous function on the domain boundary is call \textit{Dirichlet problem}, i.e. finding $H: \Omega \rightarrow \C$ such that
\begin{equation}\label{direchlet PDE}
    \begin{split}
        &\Delta H(z) = 0, \text{ } z \in \Omega,\\
        &H(z) = f(z), \text{ } z \in \partial \Omega,
    \end{split}
\end{equation}
where $\Omega$, $f: \partial \D \rightarrow \C$ are given. The existence of solution for PDE (\ref{direchlet PDE}) is proved by Hilbert in 1900 and the uniqueness can be easily checked by using the maximum principle.

However, it's still a hard job to calculate such $H$. Luckily, for a special case, where the domain is unit disk $\D$ on complex plane, the \textit{Poisson integral} shows a explicit method to obtain the desired harmonic function $H$ as
    \begin{equation}\label{poisson integral}
        H(re^{i\theta}) = \frac{1}{2\pi}\int_0^{2\pi} \frac{(1-r^2)f(e^{i \varphi})}{1 - 2 r cos (\varphi - \theta) + r^2} d\varphi.
    \end{equation}
Such $H$ is harmonic on $\D$ and continuous on $\overline{\D}$ and have the same value with $f$ on the $\partial \D$, i.e. $H(e^{i\theta}) = f(e^{i\theta})$. Of course, this $H$ is unique. Fig. \ref{harmonic} displays an example of Poisson integral.

The Poisson integral is a stable method to extend the continuous function on unit circle to a unique harmonic function on unit disk, and we call it \textit{harmonic extension}.

\begin{figure}
    \begin{center}
        \includegraphics[width=14cm]{fig4_down.png}
    \end{center}
    \caption{(a) Continuous function $f(e^{i \theta}) = \sin(10 \theta) + \cos(10 \theta) + 1.5$ defined on $\partial \D$; (b) The corresponding harmonic function $H$ generated from $f$ by the equation (\ref{poisson integral}). Note that we used absolute value to illustrate the progress of harmonic extension for the convenience of demonstration.}
    \label{harmonic}
\end{figure}


\section{Quasi-conformal mapping and Beltrami equation}
        Let $f: \Omega \subset \C \rightarrow \C$ be a complex function. The following differential operators are more convenient for discussion
        \begin{equation*}
            \frac{\partial}{\partial z} := \frac{1}{2}(\Part{}{x} - i \Part{}{y}), \Part{}{\overline{z}}:= \frac{1}{2}(\Part{}{x}+i \Part{}{y})
        \end{equation*}

        $f$ is said to be \textit{quasi-conformal} associated to $\mu$ if it oriention-preserving and satisfies the following \textit{Beltrami equation}:
        \begin{equation}\label{beltrami eq}
            \Part{f}{\overline{z}} = \mu(z) \Part{f}{z}
        \end{equation}
        where $\mu(z)$ is some complex-valued Lebesgue measurable function satisfying $\norm{\mu}_\infty < 1$. More specifically, this $\mu: \Omega \rightarrow \D$ is called the \textit{Beltrami coefficient} of $f$
        \begin{equation}
            \mu_f = \frac{f_{\overline{z}}}{f_{z}}
        \end{equation}
        In terms of the metric tensor, consider the effect of the pullback under $f$ of the Euclidean metric $ds^2_E$; the resulting metric is given by:
        \begin{equation}
            f^*(ds^2_E) = \abs{\Part{f}{z}}^2 \abs{dz + \mu(z)d\overline{z}}^2
        \end{equation}
        which, relative to the background Euclidean metric $dz$ and $d\overline{z}$, has eigenvalue $(1+\abs{\mu})^2 \abs{\Part{f}{z}}^2$ and $(1-\abs{\mu})^2 \abs{\Part{f}{z}}^2$. $\mu$ is call the \textit{Beltrami coefficient}, which is a measure of nonconformality. In particular, the map $f$ is conformal around a small neighborhood of $p$ when $\mu(p)=0$. Infinitesimally, around a point $p$, $f$ may be expressed with respect to its local parameters as follows:
        \begin{equation}\label{local f}
            f(z) = f(p)+f_z(p)(z+\mu(p)\overline{z}).
        \end{equation}

        % \begin{figure}
        %     \begin{center}
        %         \includegraphics[width=7.6cm]{fig2.jpg}
        %     \end{center}
        %     \caption{(a) Shows the original hippocampus; (b) Shows the parameter domain with circle packing pattern. Under the quasiconformal parameterization, the infinitesimal circles on the parameter domain are mapped to infinitesimal ellipses on the hippocampus, as shown in (c).}
        %     \label{fig2}
        % \end{figure}

        If $\mu(z)=0$ everywhere, then $f$ us called \textit{conformal} or \textit{holomorphic}. A conformal map satisfies the following well-known Cauchy-Riemann equation:
        \begin{equation}\label{cr eq}
            \Part{f}{\overline{z}} = 0.
        \end{equation}
        Inside the local parameter domain, $f$ may be considered as a map composed of a translation to $f(p)$ together with a stretch map $S(z) = z + \mu(p)\overline{z}$, which is postcomposed by multiplication of $f_z(p)$, which is conformal. All the conformal distortion of $S(z)$ is caused by $\mu(p)$. $S(z)$ is the map that causes $f$ to map a small circle to a small ellipse. Form $\mu(p)$, we can determine the angles of the directions of maximal magnification and shrinkage and the amount of them as well. Specially, the angle of maximal magnification is $\arg(\mu(p))/2$ with magnifying factor $1+\abs{\mu(p)}$; the angle of maximal shrinkage is the orthogonal angle $\arg(\mu(p))/2 - \pi/2$ with shrinkage factor $1-\abs{\mu(p)}$. The distortion or dilation is given by:
        \begin{equation}
            K = \frac{1+\abs{\mu(p)}}{1-\abs{\mu(p)}}.
        \end{equation}
        Thus, the Beltrami coefficient $\mu$ gives us important information about the properties of the map (see Fig. \ref{fig3}).

        \begin{figure}
            \begin{center}
                \includegraphics[width=6cm]{fig3.png}
            \end{center}
            \caption{Quasi-conformal maps infinitesimal circles to ellipses. The Beltrami coefficient measure the distortion or dilation of the ellipse under the QC map.}
            \label{fig3}
        \end{figure}

        Then we put eyes on the Beltrami coefficient of composition of two different maps. Suppose $f, g: \C \rightarrow \C$ are complex-valued function with Beltrami coefficient $\mu_f, \mu_g$ respectively. Then the Beltrami coefficient for the composition $g \circ f$ is given by 
        \begin{equation}\label{mu of composition}
            \mu_{g \circ f} = \frac{\mu_f+(\mu_g \circ f) \tau}{1+\overline{\mu_f}(\mu_g \circ f) \tau},
        \end{equation}
        where $\tau = \frac{\overline{f_z}}{f_z}$. This equation can be derived directly from the definition of Beltrami coefficient. Note that when $g$ is conformal, $\mu_g = 0$ and 
        \begin{equation}\label{mu of conformal composition}
            \mu_{g \circ f} = \mu_f.
        \end{equation}
        Conversely, when $f$ is conformal, we have $\mu_f = 0$ and
        \begin{equation}\label{mu of conformal composition 2}
            \mu_{g \circ f} = (\mu_g \circ f) \tau.
        \end{equation}
        It means similar relationship with equation (\ref{mu of conformal composition}) does not always holds, i.e. $\mu_{g \circ f} \neq \mu_g$.

        Note that there is a one-to-one correspondence between the quasi-conformal mapping f and its Beltrami coefficient $\mu$. Given $f$, there exists a Beltrami coefficient $\mu$ satisfying the Beltrami equation. Conversely, the following theorem states that given an admissible Beltrami coefficient $\mu$, there always exists an quasi-conformal mapping $f$ associating with this $\mu$.

        \begin{theorem}[Measurable Riemannian Mapping Theorem]\label{Measurable Riemannian Mapping Theorem}
            Suppose $\mu : \C \rightarrow \C$ is Lebesgue measurable satisfying $\abs{\mu}_\infty <1$; then, there exists a quasi-conformal homeomorphism $f$ from $\C$ onto itself, which is in the Sobolev space $W_{1,2}(\C)$ and satisfies the Beltrami equation in the distribution sense. The associated quasi-conformal homeomorphism $f$ is unique up to a Mobi\"us transformation. Furthermore, by fixing $0$, $1$ and $\infty$, the $f$ is uniquely determined.
        \end{theorem}

        By reflection, the above theorem can be further extended to Beltrami coefficients defined on the unit disk $\D$.

        \begin{theorem}
            Suppose $\mu : \D \rightarrow \D$ is Lebesgue measurable satisfying $\abs{\mu}_\infty <1$; then, there exists a quasi-conformal homeomorphism $f$ from $\D$ onto itself, which is in the Sobolev space $W_{1,2}(\D)$ and satisfies the Beltrami equation in the distribution sense. The associated quasi-conformal homeomorphism $f$ is unique up to a Mobi\"us transformation. Furthermore, by fixing $0$ and $1$, the $f$ is uniquely determined.
        \end{theorem}

        Above 2 theorems suggest that under suitable normalization, a homeomorphism from $\C$ or $\D$ onto itself can be uniquely determined by its associated Beltrami coefficient. We can study the way to obtain maps from Beltrami Coefficient. Let $f(x, y) = u(x, y)+iv(x, y)$. Then, by Beltrami equation (\ref{beltrami eq}), we have
        \begin{equation}\label{beltrami eq with u v}
            \mu = \frac{f_{\overline{z}}}{f_z} = \frac{(u_x-vy)+i(v_x+u_y)}{(u_x+v_y)+i(v_x-u_y)}.
        \end{equation}
        Suppose $\mu_f = \alpha + i \beta$, then we can rewrite equation (\ref{beltrami eq with u v}) and get two systems of linear equations that are:
        \begin{equation}
            \begin{cases}
                v_y = \gamma_1 u_x + \gamma_2 u_y\\
                -v_x = \gamma_2 u_x + \gamma_3 u_y
            \end{cases}
        \end{equation}
        and
        \begin{equation}
            \begin{cases}
                u_y = \gamma_1 v_x + \gamma_2 v_y\\
                -u_x = \gamma_2 v_x + \gamma_3 v_y
            \end{cases}
        \end{equation}
        where
        \begin{equation}
            \gamma_1 = \frac{(1-\alpha)^2 + \beta^2}{1-\alpha^2-\beta^2}, \gamma_2 = -\frac{2\beta}{1-\alpha^2-\beta^2}, \gamma_3 = \frac{(1+\alpha)^2 + \beta^2}{1-\alpha^2-\beta^2}.
        \end{equation}
        Since 
        \begin{equation}
            \begin{split}
                &\left\langle \nabla ,  \begin{pmatrix}
                    v_y \\ -v_x
                \end{pmatrix} \right\rangle = v_{xy} -v_{xy} = 0 \\
                &\left\langle \nabla ,  \begin{pmatrix}
                    u_y \\ -u_x
                \end{pmatrix} \right\rangle = u_{xy} -u_{xy} = 0,
            \end{split}
        \end{equation}
        it yields
        \begin{equation}\label{LBS}
            \begin{split}
                &\left\langle \nabla ,  A\begin{pmatrix}
                    v_y \\ -v_x
                \end{pmatrix} \right\rangle = v_{xy} -v_{xy} = 0 \\
                &\left\langle \nabla ,  A\begin{pmatrix}
                    u_y \\ -u_x
                \end{pmatrix} \right\rangle = u_{xy} -u_{xy} = 0,
            \end{split}
        \end{equation}
        with $A = \begin{pmatrix}
            \gamma_1 & \gamma_2\\\gamma_2 & \gamma_3
        \end{pmatrix}$. It is obvious that $\det(A) > 0$ and $A$ is symmetric, so A is symmetric positive definite. It has been shown that a quasiconformal map could be uniquely determined up to Mobi\"us transformation in theorem \ref{Measurable Riemannian Mapping Theorem}, and Mobi\"us transformation can be fixed by three points. It means that we need only three points as boundary condition to solve the systems of equation (\ref{LBS}). Indeed, this process was invented by Lui \etal \cite{lui2012beltrami} which is called the Linear Beltrami Solver(LBS).
























\chapter{Conclusion and Future works} \label{conclusion}

In this thesis, we propose a novel shape representation for 2D bounded simply-connected objects called Harmonic Beltrami signature. The proposed signature is based on conformal welding but overcome a key shortcoming that it can be uniquely determined by the given shape. What's more exciting is that the proposed representation is invariance under scaling, translation and rotation. For slight deformation and distortion, HBS keeps robust and only changes within a reasonable small range. Therefore, there are reasons to believe the it does have ability to represent some invariant geometrical features. The experimental results also confirm that the HBS has excellent performance in multi-classification tasks.

Although our work has achieved relatively good results, the proposed representation still have some limitations. Firstly, the HBS is only applicable to simply-connected shapes currently, but as a matter of fact, multi connected images are the majority in the real world. So we are eager for a feasible method to extend our HBS to multi connected situation. Secondly, the traditional algorithm to compute the Beltrami coefficient is inevitably dependent on triangular mesh, which consumes a lot of time. Therefore, a fast algorithm to obtain this signature avoiding dense mesh is of high priority in our future work.

In summary, we will focus on three major directions in the future. One is that the deeper meaning of HBS is worth digging and then a multi-connected version of representation based on this work can be proposed. Another is that if the HBS contains some geometrical features of shapes, we can also extract them directly from images and generate the HBS again. Hence the deep learning theory may help us to compute this signature from given images immediately, which is very likely to improve algorithm speed performance greatly. A third direction is this representation can be used in more applications like segmentation, registration and so on.
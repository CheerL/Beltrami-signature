\chapter{Experimental result}\label{result}
        In this section, we will validate key properties of our proposed Harmonic Beltrami signature, the invariance of under simple transformations and the robustness under small distortion and modification. Besides that, a good shape representation should keep the similarity with the same kind shape and is significantly different from different kinds of shape.

        Before showing results, what needs illustration is that the distance we used to measure the difference of HBS is based on $L^2$ norm,
        \begin{equation}\label{bs dis}
            d(B_1, B_2) = \sqrt{\frac{1}{N} \sum_{i=1}^N \abs{B_1(z_i) - B_2(z_i)}^2},
        \end{equation}
        where $B_1, B_2$ are two different Harmonic Beltrami signatures, $z_i \in \D$ is the face center of triangular mesh $M$ mentioned in Section \ref{detail harmonic} and $N=60962$ here.
        \section{Invariance}\label{sec inv}
            We use the dolphin shown in Fig \ref{illu of BS} (a) as the original shape, then calculate HBS after scaling, translation and rotation and compare them with the original shape's HBS. The result is displayed in Fig \ref{inv}. In this figure, the first column are the sets of boundary points and we remark them as $\Omega_a$ to $\Omega_f$. The second column are the corresponding Harmonic Beltrami signatures $B_a$ to $B_f$. Note that all the Harmonic Beltrami signatures are shown in modulus, i.e. $\abs{B_n}$ for row $n$, and in top view. And the third column(if have) are the histograms of the difference between original shape's Harmonic Beltrami signature, i.e. $\abs{B_n - B_a}$ for row $n$.
            
            Row b and c are about scaling, the shapes are $\Omega_b = \{z \mid z = 1.5 z_a, z_a \in \Omega_a \}$ and $\Omega_c = \{z \mid z = 0.5 z_a, z_a \in \Omega_a \}$ and the distance are $d(B_a, B_b) = 5.5647 \times 10^{-8}$ and $d(B_a, B_c) = 5.3476 \times 10^{-8}$. Row d is about translation, the shape $\Omega_d = \{z \mid z = z_a+100+20i, z_a \in \Omega_a \}$ and the distance is $d(B_a, B_d) = 4.7817 \times 10^{-8}$. Row e is about rotation, the shape is $\Omega_e = \{z \mid z = e^{0.2\pi i} z_a, z_a \in \Omega_a \}$ and the distance is $d(B_a, B_e) = 5.2144 \times 10^{-8}$. Row f is the combination of scaling, translation and rotation, the shape is $\Omega_e = \{z \mid z = 3e^{-0.85\pi i} z_a+350+600i, z_a \in \Omega_a \}$ and the distance is $d(B_a, B_e) = 5.7635 \times 10^{-8}$. These confirm the invariance of HBS under scaling, translation and rotation.

            \begin{figure}
                \begin{center}
                    \includegraphics[width=11cm]{inv.png}
                \end{center}
                \caption{Harmonic Beltrami signature under scaling, translation and rotation.}
                \label{inv}
            \end{figure}

        \section{Robustness}
            Similar with Section \ref{sec inv}, here we still treat the dolphin as the original shape and modify some small parts of it and Fig \ref{robust} is the result. It shows that the proposed signature is robust and stable and will not have a big mutation caused by small disturbance.

            Row g, h and i are result about modification. These shapes are generated by removing or adding something, which is in the red circle. We can see that Harmonic Beltrami signatures have slight differences from $B_a$ but are still similar in general. And this figure also demonstrates that the bigger the modification part is, the more different the Harmonic Beltrami signature is. For example in row i, losing a half of the tail makes the signature has a marked change. Quantitatively, $d(B_a, B_g) = 0.0132$, $d(B_a, B_h) = 0.0316$ and $d(B_a, B_i) = 0.1761$.

            Row j is for distortion. This dolphin is only enlarged horizontally and becomes fatter, then the $B_j$ moves a little bit and $d(B_a, B_j) = 0.0724$.

            \begin{figure}
                \begin{center}
                    \includegraphics[width=11cm]{robust.png}
                \end{center}
                \caption{Harmonic Beltrami signature under small modification}
                \label{robust}
            \end{figure}

        \section{Classification with HBS}
            Above properties ensure the proposed signature having the ability to reflect some stable features of given shape, but another much more important thing people concerned is that whether it can distinguish a shape from many different kinds of shapes and classify it correctly.

            To compare the classification performance, we also use conformal welding directly to classify, and the distance is defined as
            \begin{equation}\label{welding dis}
                d_c(f_1, f_2) = \sqrt{\frac{1}{N} \sum_{i=1}^N \abs{f_1(z_i) - f_2(z_i)}^2},
            \end{equation}
            where $f_1, f_2$ are two different conformal welding, $N=200$ here.

            We prepare 3 kinds of animals, fish, giraffe and elephant. There are 3 images for each group so 9 images in total. From Fig. \ref{classification images}, we can find that each class share similar HBS and conformal welding, where the top left is the input shape,  bottom left is the conformal welding and the right is Harmonic Beltrami signature in each subfigure. 
            
            The distances between each two shapes are as shown in Fig. \ref{dis matrix BS} and Fig. \ref{dis matrix welding}. The intraclass distance of HBS is always less than 0.2 while the interclass distance is greater than 0.2. But for conformal welding, the data is messy, for example, fish 3 thinks itself is very different from other fishes but looks most like giraffe 2. After multidimensional scaling(MDS), we can maps all these 9 shapes to points on 2D plane as Fig \ref{mds1}, where the HBS shows powerful classification ability. For HBS method, points representing shapes of the same kind are closely gathered together while there is no such phenomenon for conformal welding.

            \begin{figure}
                \begin{center}
                    \includegraphics[width=\textwidth]{figs.png}
                \end{center}
                \caption{These 3 rows are elephant, fish and giraffe. In each subfigure, the top left is the input shape, bottom left is the conformal welding and the right is Harmonic Beltrami signature.}
                \label{classification images}
            \end{figure}

            \begin{figure}
                \begin{center}
                    \includegraphics[width=14cm]{distance.png}
                \end{center}
                \caption{(a) The distance matrix of Harmonic Beltrami signatures of above 9 shapes by equation (\ref{bs dis}); (b) The distance matrix of conformal weldings of above 9 shapes by equation (\ref{welding dis}).}
                \label{dis matrix welding}
            \end{figure}

            \begin{figure}
                \begin{center}
                    \includegraphics[width=12cm]{mds_down.png}
                \end{center}
                \caption{(a) The MDS result of Harmonic Beltrami signature; (b) The MDS result of conformal welding.}
                \label{mds1}
            \end{figure}
            
        \section{Classification for more classes}\label{classification for more shapes}
            In the last experiment, the numble of samples is only 9 in total and may be too small to be convincing. So here we enlarge the amount of images to 58 with 7 different classes, which are camel, deer, dog, elephant, giraffe, gorilla and rabbit. All these shapes are in Fig. \ref{all images}.
            
            We calculate the distance matrix by equation (\ref{bs dis}) and (\ref{welding dis}) and use MDS to remap these shapes to 2D plane accordingly, then $k$-medoids method is used to cluster these points to 7 classes. The MDS and clustering results based on Harmonic Beltrami signature and conformal welding are displayed in Fig. \ref{mds_and_cluster_BS}. Not expectedly, our HBS has excellent classification performance: the classification accuracy based on Harmonic Beltrami signature is 94.83\%, while the accuracy based on conformal welding is only 37.93\%.
            
            \begin{figure}
                \begin{center}
                    \includegraphics[width=15cm]{all_images.png}
                \end{center}
                \caption{All 58 shapes within 7 classes used in experiment \ref{classification for more shapes}.}
                \label{all images}
            \end{figure}

            \begin{figure}
                \begin{center}
                    \includegraphics[width=12cm]{mds_and_cluster_BS.png}
                \end{center}
                \caption{(a) The MDS result of Harmonic Beltrami signature; (b) The clustering result of Harmonic Beltrami signature.}
                \label{mds_and_cluster_BS}
            \end{figure}

            \begin{figure}
                \begin{center}
                    \includegraphics[width=12cm]{mds_and_cluster_welding.png}
                \end{center}
                \caption{(a) The MDS result of conformal welding; (b)  The clustering results of conformal welding.}
                \label{mds_and_cluster_welding}
            \end{figure}
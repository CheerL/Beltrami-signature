\chapter{Implementation detail}\label{implementation}
\section{Zipper algorithm}
    In order to find a unique and stable HBS, the first thing is to do find a way to calculate a conformal mapping from the given domain to unit disk. As mentioned in Section \ref{norm1}, we only have finite boundary points of the shape and zipper algorithm invented in the 1980s is a suitable and accurate method to deal with this situation numerically. 

    Marshall \etal demonstrates the zipper algorithm detailedly with clear diagrams in \cite{marshall2007convergence}. For the convenience of readers, we gives a very brief review here. Given $N$ clockwise boundary points $z_1, z_2, \cdots, z_N \in \partial \Omega$, this algorithm use a series of linear fractional transformations $g_1, g_2, \cdots, g_N$ to map $z_1, z_2,\cdots, z_N$ to real axis one-by-one, and finally transform the upper half plane to unit disk by $g_{N+1}(z) = \frac{z-i}{z+i}$. Therefore, $g = g_{N+1} \circ g_N \circ \cdots \circ g_2 \circ g_1$ is a conformal mapping indisputably and maps all these boundary points to unit circle and the domain $\Omega$ to $\D$. Remark that zipper algorithm is sensitive to the order of points. If we input the points anti-clockwise, i.e. $z_N, z_{N-1}, \cdots, z_1$, the zipper will give us a conformal mapping from $\Omega^c$ to $\D$. This progress is shown in Fig. \ref{zipper algo}.

    \begin{figure}
    \begin{center}
        \includegraphics[width=15cm]{zipper_algo.png}
    \end{center}
    \caption{Zipper algorithm}
    \label{zipper algo}
    \end{figure}

    For $\Phi_1 : \D \rightarrow \Omega$, we can find a conformal mapping $g_{\Phi_1}: \Omega \rightarrow \D$ by inputting points clockwise, and $\Phi_1 = g_{\Phi_1}^{-1}$. For $\Phi_2 : \D^c \rightarrow \Omega^c$, we can input anti-clockwise points and get $g_{\Phi_2} : \Omega^c \rightarrow \D$. At the result $\Phi_2(z) = g_{\Phi_2}^{-1}(\frac{1}{z})$. Because of the invariance of scaling, the number of boundary points $N$ is fixed as $200$ here and they are picked uniformly from the shape contour. 

    \begin{algorithm}[H]                           % HERE!!!!!!!!!
    \caption{Zipper}          % give the algorithm a caption
    \label{zipper}      % and a label for \ref{} commands later in the document
    \begin{algorithmic}  % enter the algorithmic environment
        \STATE \textbf{Inputs:} $z_i \in \partial \Omega$ for $i=1, 2, \cdots, N$, $N=200$. %, $\epsilon = 10^{-10}$(any very small value)
        \STATE \textbf{Initialize:} Let $r=2$, $g_1(z) = \sqrt{\frac{z-z_2}{z-z_1}}$, $g= g_1$ and compute $p_{i,2}=g(z_i)$.
        \WHILE{$r < N$}
            \STATE Pick $q = p_{r+1, r} = a + b i$, then compute $c = \frac{a}{\abs{q}^2}$, $d = \frac{b}{\abs{q}^2}$.
            \STATE Let $g_r(z) = \sqrt{\frac{cz}{1+dzi}}$, then $g = g_r \circ g$ and compute $p_{i, r+1} = g_r(p_{i, k})$.
            \STATE Let $r = r+1$.
        \ENDWHILE
        \STATE Let $g_{N}(z) = \left(\frac{z}{1-\frac{z}{p_{1,N}}}\right)^2$ and $g_{N+1}(z) = \frac{z-i}{z+i}$, then $g = g_{N+1} \circ g_N \circ g$ and $p_i = g_{N+1} \circ g_N(p_{i,N})$.
        \RETURN Conformal mapping $g: \Omega \rightarrow \D$ and boundary points $p_i \in \partial \D$.
    \end{algorithmic}
    \end{algorithm}
    
\section{Harmonic extension}\label{detail harmonic}
    After obtaining the output of zipper algorithm, $\Phi_1(z) = g_{\Phi_1}^{-1}(z)$ and $\Phi_2(z) = g_{\Phi_2}^{-1}(\frac{1}{z})$, the conformal welding can be represented as a series points 
    \begin{equation*}
        (\varphi_i, \omega_i) = \left(\arg(\hat{\Phi}_2^{-1}(z_i)), \arg(\hat{\Phi}_1^{-1}(z_i))\right),
    \end{equation*}
    where $\varphi_i, \omega_i \in [0, 2\pi)$. So in fact we should use discrete form Poisson integral to extend $f$ to a harmonic mapping $H$ on the unit disk
    \begin{equation}\label{discrete poisson integral}
        H(re^{i\theta}) = \frac{1}{2\pi} \sum_{j=1}^k \frac{(1-r^2) e^{i \omega_j} \gamma_j}{1 - 2 r cos (\varphi_j - \theta) + r^2} ,
    \end{equation}
    where $\gamma_j = (\varphi_{j} - \varphi_{j-1}) \bmod{ 2\pi}$ and $\varphi_0 = \varphi_k$ and this $\bmod$ can solve some critical value problem, for example, $\varphi_j = 0$ but $\varphi_{j-1} = 6$. For the convenience of computation, we only calculate the value of $H$ on a grid
    \begin{align*}
        G = \{z=x+iy \mid &\abs{z} \le 1, x=\frac{j}{100}, y=\frac{k}{100},\\
        &j,k=-100,-99,\cdots,99,100 \}.
    \end{align*}

\section{Normalization}
    Because of the arbitrariness of the conformal mapping, two different harmonic extension $H$ and $\tilde{H}$ of the given domain $\Omega$ have relationship
    \begin{equation*}
        \tilde{H} = M_1 \circ H \circ M_2.
    \end{equation*}
    Section \ref{norm1} and \ref{norm2} show that we can normalize $M_1$ and $M_2$ by some restrictions and then HBS $B$ can be unique. This section will tells how to satisfy equation (\ref{norm phi1}) and (\ref{norm phi2 1}) from the output of zipper algorithm.

    To normalization $M_1$, we need to solve equation (\ref{eq}). Generally speaking, the output of zipper, $p_i = g_{\Phi_1}(z_i) \in \D, i = 1, 2, \cdots, N$, will concentrate around a point. At that time, $\abs{f(a)} \approx 1$ almost everywhere and the solution of (\ref{eq}) is also very close to the point. This means the solution is quite unstable and hard to converge for common algorithms(see Fig \ref{original distribution}).

    \begin{figure}
        \begin{center}
            \includegraphics[width=6cm]{fig7.png}
        \end{center}
        \caption{(a) The $p_i \in \partial \D$ gather in a small neighborhood around their arithmetic mean $p_c$, which is labeled in red; (b) The corresponding $\abs{f(a)}$ for these $p_i$. It's worth to mention that the minimal of $\abs{f(a)}$ can reach actually, but it isn't shown in the picture since the grid is not small enough.}
        \label{original distribution}
    \end{figure}

    \begin{figure}
        \begin{center}
            \includegraphics[width=6cm]{fig8.png}
        \end{center}
        \caption{Similar with Fig \ref{original distribution}. (a) The boundary points after adjustment; (b) The $\abs{f(a)}$.}
    \end{figure}

    Instead of proposing a complicated method to solve equation directly, the solution we adopted to solve this problem is to use some Mobi\"us transformations to adjust the distribution of $p_{i}$ until it becomes almost uniform. For the $r$-th iteration, let 
    \begin{equation*}
        p_{c,r} = \frac{\sum_{i=1}^k p_{i,r}}{k} \in \D
    \end{equation*}
    as the arithmetic center of $p_{i,r} \in \partial \D$. Remark that $F_a(z) = \frac{z-a}{1-\overline{a}z}$ is a Mobi\"us transformation ignoring rotation, then the $F_{p_{c,r}}$ gives new boundary points on unit disk as 
    \begin{equation*}
        p_{i,r+1} = F_{p_{c,r}}(p_{i,r}) = \frac{p_{i,r} - p_{c,r}}{1 - \overline{p_{c,r}}p_{i,r}}.
    \end{equation*}
    Note that we set $p_{i,0} = p_{i}$ at the beginning. Repeat this iteration for $r$ times until the center of boundary points is close to $0$, then the distribution is sufficiently regular and $\Phi_1$ becomes
    \begin{equation}
        \tilde{\Phi}_1 = \Phi_1 \circ F_{p_{c,0}}^{-1} \circ F_{p_{c,1}}^{-1} \circ \cdots \circ F_{p_{c,r}}^{-1}.
    \end{equation}
    Now equation (\ref{eq}) can be solved without much effort by Newtown's method. Suppose $a \in \D$ is the optimal solution and let 
    \begin{equation*}
        M_{\Phi_1} = F_a \circ F_{p_{c,r}} \circ F_{p_{c,r-1}} \circ \cdots \circ F_{p_{c, 0}}
    \end{equation*}
    and the final conformal mapping satisfying (\ref{norm phi1}) is
    \begin{equation}
        \hat{\Phi}_1 = \tilde{\Phi}_1 \circ F_{a}^{-1} = \Phi_1 \circ M_{\Phi_1}^{-1}
    \end{equation}

    \begin{algorithm}[H]
    \caption{Normalize $M_1$}
    \label{alg norm phi1}
    \begin{algorithmic}
        \STATE \textbf{Inputs:} $\Phi_1$ and $p_i \in \partial \D$ for $i=1, 2, \cdots, N$, $N=200$, $\epsilon=0.2$.
        \STATE \textbf{Initialize:} Let $r=0$, $p_{i,0} = p_i$, $M_{\Phi_1} = id$ and compute $p_{c,0}= \frac{1}{N} \sum_{i=1}^N p_{i}$. %and $M_{p_{c,0}}(z) = \frac{z-p_{c,0}}{1- \overline{p_{c, 0}}z}$
        \WHILE{$\abs{p_c,r} > \epsilon$}
            \STATE Let $F_{p_{c,r}}(z) = \frac{z-p_{c,r}}{1- \overline{p_{c, r}}z}$ and $M_{\Phi_1} = F_{p_{c,r}} \circ M_{\Phi_1}$.%and $\Phi_1 = \Phi_1 \circ F_{p_{c,r}}^{-1}$.
            \STATE Compute $p_{i, r+1} = F_{p_{c,r}}(p_{i, r})$ and $p_{c,r+1} = \frac{1}{N} \sum_{i=1}^N p_{i, r+1}$.
            \STATE Let $r = r+1$.
        \ENDWHILE
        \STATE Solve equation (\ref{eq}) by Newtown's method and get solution $a \in \D$.
        \STATE Let $F_a(z) = \frac{z-a}{1- \overline{a}z}$ and $M_{\Phi_1} = F_a \circ M_{\Phi_1}$.% and $\hat{\Phi}_1 = \Phi_1 \circ F^{-1}_{a}$.
        %\STATE Compute $p_i = F_a(p_{i,r})$.
        \RETURN Mobi\"us transformation $M_{\Phi_1}$.%, conformal mapping $\hat{\Phi}_1: \D \rightarrow \Omega$ satisfying (\ref{norm phi1}) and boundary points $p_i \in \partial \D$.
    \end{algorithmic}
    \end{algorithm}

    \begin{figure}
        \begin{center}
            \includegraphics[width=10cm]{fig9.png}
        \end{center}
        \caption{The iteration of distribution adjustment. The boundary points are blue and their arithmetic center is red in each picture. (a)-(d) The 1st, 3rd, 7th, 12th iteration.}
    \end{figure}
    
    As for the normalization of $M_2$, it's much easier. For requirement (\ref{norm phi2 1}), let $b = \Phi_2^{-1}(\infty)$ and $c = -\frac{1}{\bar{b}}$, from
    \begin{equation*}
        F_{c}(\infty) = \lim_{z \rightarrow \infty} \frac{z -c}{1 - \bar{c}z} = -\frac{1}{\overline{c}} = b
    \end{equation*} we have 
    \begin{equation*}
        \tilde{\Phi}_2(\infty) = \Phi_2 \circ F_{c}(\infty) = \Phi_2(b) = \infty.
    \end{equation*}
    So we only need replace $\Phi_2$ with $\tilde{\Phi}_2 =\Phi_2 \circ F_{c}$ to map $\infty$ to $\infty$.

    % According to zipper algorithm, we know each linear fractional transformation composing $\Phi_2$, and the derivative of $\tilde{\Phi}_2$ can be found by the chain derivative rule. And another requirement (\ref{norm phi2 2}) can be satisfied by finding $a=\argmax{z \in \partial\D} \abs{\tilde{\Phi}_2'(z)}$.  Here we select $M = 10^5$ points uniformly from $[0, 2\pi)$ as $\theta_j = \frac{2\pi}{M}(j-1)$ and calculate the $\abs{\tilde{\Phi}_2'(e^{i \theta_j})}$ (see Fig \ref{derative}). 
    
    % Therefore, the maximal solution
    % \begin{equation}
    %     \theta_{\max} = \argmax{j \in [1,10^5]} \abs{\tilde{\Phi}_2'(e^{i \theta_j})}
    % \end{equation}
    % can be used to normalize the rotation and the final mapping is 
    % \begin{equation}
    %     \hat{\Phi}_2(z) = \tilde{\Phi}_2(e^{i \theta_{\max}}z) = \Phi_2 \circ M_{\Phi_2}(z),
    % \end{equation}
    % where $M_{\Phi_2} = e^{i \theta_{\max}} F_{ce^{-i \theta_{\max}}}$.

    \begin{algorithm}[H]
    \caption{Normalize $M_2$}
    \label{alg norm phi2}
    \begin{algorithmic}
        \STATE \textbf{Inputs:} $\Phi_2$ and $p_i \in \partial \D$ for $i=1, 2, \cdots, N$, $N=200$, $M=10^5$.
        \STATE Compute $b = \overline{\Phi_2^{-1}(\infty)}$ and $c = -\frac{1}{\bar{b}}$.
        \STATE Let $F_c(z) = \frac{z-c}{1-\overline{c}z}$ and $M_{\Phi_2} = F_{c}$.
        \RETURN Mobi\"us transformation $M_{\Phi_2}$
        % \RETURN Conformal mapping $\hat{\Phi}_2: \D^c \rightarrow \Omega^c$ satisfied (\ref{norm phi2 1}), (\ref{norm phi2 2}) and boundary points $p_i \in \partial \D$.
    \end{algorithmic}
    \end{algorithm}

    After obtaining Mobi\"us transformations $M_{\Phi_1}$ and $M_{\Phi_2}$, the normalized harmonic extension is
    \begin{align*}
        \hat{H} = M_{\Phi_1} \circ H \circ M_{\Phi_2}
    \end{align*}
    and it is unique up to rotation. Finally, the desired HBS can be generated from $\mu_{\hat{H}}$ according to equation (\ref{normaled B}).
    

\section{Summary of the Algorithm}
    The totally algorithm is as following.
    \begin{algorithm}[H]
    \caption{Calculate HBS}
    \label{alg all}
    \begin{algorithmic}
        \STATE \textbf{Inputs:} Simply-connected shape $\Omega \subset \C$, $N=200$.
        \STATE Pick clockwise points $z_1, \cdots, z_N \in \partial \Omega$ uniformly.
        \STATE Input $z_1, z_2, \cdots, z_N$ to Algorithm \ref{zipper} and get conformal mapping $g_{\Phi_1}: \D \rightarrow \Omega$ and $p_{i,1} \in \partial\D$.
        \STATE Input $z_N, \cdots, z_2, z_1$ to Algorithm \ref{zipper} and get conformal mapping $g_{\Phi_2}: \D \rightarrow \Omega^c$ and $p_{i,2} \in \partial\D$.
        \STATE Let $\Phi_1(z) = g_{\Phi_1}^{-1}(z)$ and $\Phi_2(z) = g_{\Phi_2}^{-1}(\frac{1}{z})$.
        \STATE Let $f = \Phi_1^{-1} \circ \Phi_2$ and represent it by $(\varphi_i, \omega_i) = (\arg(p_{i,2}), \arg(p_{i,1}))$.
        \STATE Extend $f$ to $H$ on $\D$ by equation (\ref{discrete poisson integral}) on grid $G$.
        \STATE Input $\Phi_1$ and $p_{i,1}$ to Algorithm \ref{alg norm phi1}, then get $M_{\Phi_1}$.
        \STATE Input $\Phi_2$ and $p_{i,2}$ to Algorithm \ref{alg norm phi2}, then get $M_{\Phi_2}$.
        \STATE Calculate normalized harmonic extension $\hat{H} = M_{\Phi_1} \circ H \circ M_{\Phi_2}$.
        \STATE Calculate Beltrami coefficient $\mu_{\hat{H}}$.
        \STATE Calculate $\theta = \arg \int_\D \mu_{\hat{H}}(z) dz$.
        \STATE Let $B(z) = e^{i \theta} \mu_{\hat{H}}(e^{-\frac{1}{2}i\theta} z)$.
        \RETURN Harmonic Beltrami signature $B$.
    \end{algorithmic}
    \end{algorithm}